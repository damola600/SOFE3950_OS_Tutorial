% Created 2018-02-22 Thu 16:55
% Intended LaTeX compiler: pdflatex
\documentclass[12pt]{article}
\usepackage[utf8]{inputenc}
\usepackage[T1]{fontenc}
\usepackage{graphicx}
\usepackage{grffile}
\usepackage{longtable}
\usepackage{wrapfig}
\usepackage{rotating}
\usepackage[normalem]{ulem}
\usepackage{amsmath}
\usepackage{textcomp}
\usepackage{amssymb}
\usepackage{capt-of}
\usepackage{hyperref}
\author{Justin Kaipada}
\date{\today}
\title{}
\hypersetup{
 pdfauthor={Justin Kaipada},
 pdftitle={},
 pdfkeywords={},
 pdfsubject={},
 pdfcreator={Emacs 25.3.2 (Org mode 9.1.6)},
 pdflang={English}}
\begin{document}



\begin{center}
  \section*{\textbf{SOFE 3950U: Tutorial 5 Activity}}
  \label{sec:org9d54f69}
\end{center}

\begin{center}
  \textbf{Group 1}
\end{center}

\begin{center}
  \includegraphics[scale=1.5]{uoit_logo.png}
\end{center}

\vspace{50mm}

\begin{center}
  \textbf{Justin Kaipada 10059016}\\
  \textbf{Anthea Ariyajeyam 100556294}
\end{center}

\newpage

\section*{Conceptual Question's Answers}
\label{sec:org2822a7e}
\begin{enumerate}
\item \texttt{pthread\_create} - Will create a new thread based in the \emph{POSIX}
standards using the given function as an entry point.

\texttt{pthread\_join} - Ask the calling thread to wait until the given
thread terminates, essentially joining the current thread with the
child/sub thread.

\texttt{pthread\_exit} - This function terminates the calling thread, it
makes the passed pointer available to any successful join
functions with the terminating thread

\item While threads have there on stack memory to store independent
information, all the threads created by a process share the same
heap memory and data arrays. This means that all the variables
local and global are shared by different threads if they are
created by the same process.

\item In \textbf{multithreading} many threads are created by a process to do work
concurrently and/or parallel.  In \textbf{multiprocessing} instead of
creating multiple threads, the program created many processes to
the work.

Both have pros and cons of there own. Since threads share the same
data and variables when in the same process, communication b/w
threads are easy, we only need to synchronize them carefully. But
for processes the data and variables are never shared so we need
to use some sort of shared memory system or messaging system which
will make them slower. If we use a cluster system to run our
programs and we do not use shared memory, multiprocessing is the
way to go because, each process can be run on different nodes in a
cluster but multithreading does not enable us to do that. One of
main problem when using multiprocessing to do simple work is the
overhead of creating a process, which is high in both CPU usage
and memory.

\item \textbf{Mutual exclusion} is the act of preventing simultaneous access of
the shared data/streams between threads, by using a mutex or some
other mechanisms like a mutex. In fact, MUTEX is an abbreviation
for MUTual EXclusion.

When using multiple threads that share the same data, there will
be the some section of code which is where we modify the data or
access input or output streams, these sections of code are called
critical sections which should be done atomically. We will use a
mutex to do this, by locking the mutex before the critical section
and unlocking it after we finish accessing/modifying the data in
the critical section.

\item The methods used to perform mutual exclusion when using pthreads are

\texttt{pthread\_mutex\_init (mutex,attr)} - This method is used to
dynamically initialize the \texttt{mutex} with the attributes specified as
\texttt{attr}.

\texttt{pthread\_mutex\_destroy (mutex)} - This method is used to
destroy/free the \texttt{mutex} object when it is no longer needed.

\texttt{pthread\_mutexattr\_init (attr)} - This method is used to create
a mutex \texttt{attr} which is needed for the creation of a new mutex.

\texttt{pthread\_mutexattr\_destroy (attr)} - This method is used to
destroy/free the \texttt{attr} object when we no longer need it.

\texttt{pthread\_mutex\_lock (mutex)} - This method is used to acquire
lock the specified \texttt{mutex} so the critical data can be accessed
without interference from other threads

\texttt{pthread\_mutex\_trylock (mutex)} - This will try to lock a \texttt{mutex}
and return and error code if the mutex is already locked by
another thread instead of waiting it to be unlocked preventing
deadlocks.

\texttt{pthread\_mutex\_unlock (mutex)} - This will unlock the \texttt{mutex} if
called by the owning thread. This will enable other threads to
lock and use the critical data.
\end{enumerate}
\end{document}